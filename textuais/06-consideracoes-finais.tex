\mychapter{Considerações Finais}
\label{Cap:ConsideracoesFinais}

O desenvolvimento da plataforma DDR foi bem-sucedido, atingindo a maior parte dos objetivos inicialmente propostos, especialmente no que se refere à melhoria da consistência e qualidade dos dados industriais por meio da aplicação de métodos avançados como os multiplicadores de Lagrange. O uso de tecnologias modernas, como \textit{React.js}, \textit{Python} e \textit{ReactFlow}, foi fundamental para a criação de uma interface \textit{web} escalável, eficiente e intuitiva, adequada para o monitoramento e controle de processos industriais em tempo real e a interface final do cliente foi capaz de proporcionar uma visualização clara e interativa dos fluxos de dados industriais.

Para o futuro, a plataforma DDR tem grande potencial para expansão e melhorias. A incorporação de ferramentas de \textbf{predição de comportamento}, com o uso de \textit{machine learning}, poderia proporcionar \textit{insights}, permitindo a detecção antecipada de falhas, otimização de manutenção preventiva e identificação de padrões anômalos nos dados reconciliados. Esse avanço agregaria um valor significativo à plataforma, transformando-a em uma ferramenta proativa na gestão de processos industriais.

Além disso, aprimoramentos na interface, como a adição de \textbf{ferramentas de análise avançada}, gráficos interativos e \textit{dashboards} personalizáveis com alertas automáticos, poderiam melhorar ainda mais a usabilidade e eficiência do sistema. Esses recursos facilitariam o monitoramento contínuo, permitindo respostas rápidas a condições críticas e melhorando a eficácia operacional.

Outro ponto importante seria a expansão da integração com sistemas de gestão como ERP e SCADA. A comunicação direta entre os dados reconciliados e as plataformas de gestão estratégica permitiria uma maior automação e sincronização dos processos industriais, otimizando a tomada de decisões.
Outro ponto importante seria a expansão da integração com sistemas de gestão como ERP e SCADA. A comunicação direta entre os dados reconciliados e as plataformas de gestão estratégica permitiria uma maior automação e sincronização dos processos industriais, otimizando a tomada de decisões.

Com essas melhorias, o DDR tem o potencial de se consolidar como uma ferramenta indispensável na \textit{Indústria 4.0}, não apenas facilitando a reconciliação de dados, mas também oferecendo \textbf{insights} preditivos e maior automação. Assim, a plataforma se posiciona como uma base sólida para inovações futuras no setor industrial, contribuindo para o avanço tecnológico e o aumento da competitividade das indústrias.
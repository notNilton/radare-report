\mychapter{Apresentação da Inovação Científica}
\label{Apresentacao}

No Capítulo \Ref{Apresentacao} será feita a apresentação do trabalho produzido, introduzindo os fundamentos da solução proposta, aqui denominada \textbf{DDR} (\textit{Dashboard of Data Reconciliation}), que visa proporcionar uma plataforma inovadora para a reconciliação e análise de dados industriais, com ênfase na facilidade de manuseabilidade e adaptação à ferramenta DDR.

\section{Problematização}

A reconciliação de dados em processos industriais no Brasil sempre representou um desafio para a garantia de qualidade e confiabilidade das informações coletadas \cite{datarecshakar}. O setor industrial brasileiro frequentemente enfrenta imprecisões em dados provenientes de sensores e equipamentos utilizados em plantas industriais \cite{produtividadeindustria}. Tais imprecisões, derivadas de erros aleatórios ou sistemáticos nas medições, comprometem a eficiência operacional e dificultam a tomada de decisões estratégicas \cite{datarecsurvey}. A ausência de ferramentas ágeis e acessíveis, capazes de realizar a reconciliação de dados em tempo real ou em bateladas, impacta negativamente a competitividade das indústrias nacionais \cite{industry4status}.

Além disso, há uma notável carência no mercado nacional de soluções tecnológicas que integrem processos industriais complexos de forma eficaz, combinando uma interface de fácil uso com a capacidade de processamento de dados em larga escala \cite{danielhoduin}. As soluções disponíveis no mercado brasileiro frequentemente apresentam custos elevados e dependem de infraestrutura local robusta, tornando-se inviáveis para pequenas e médias empresas \cite{industryinternet}. 

Portanto, a inovação tecnológica apresentada visa superar essas limitações por meio de uma plataforma \textit{online} acessível, eficiente e escalável, projetada para atender às necessidades específicas das indústrias brasileiras e contribuir para seu desenvolvimento e crescimento sustentável \cite{reconset}.

\section{Metas}

A inovação teve como principais metas, cada uma delas detalhada a seguir para esclarecer as direções e objetivos específicos do projeto:

\subsection{Desenvolver uma ferramenta online}

O desenvolvimento de uma ferramenta inteiramente \textit{online} foi um dos pilares centrais do DDR \cite{bilco}. O objetivo dessa meta foi garantir que o sistema pudesse ser acessado remotamente, sem a necessidade de instalação de \textit{software} ou infraestrutura física complexa. Ao optar por uma solução baseada na \textit{web}, o DDR oferece alta acessibilidade para diversas indústrias, independentemente da localização geográfica ou dos recursos tecnológicos disponíveis localmente \cite{industrynew}. A utilização de tecnologias \textit{web} modernas permite que o sistema funcione de maneira eficaz em uma variedade de dispositivos e plataformas, garantindo facilidade de acesso e manutenção simplificada \cite{javascriptframework}. Essa abordagem também reduz custos operacionais, já que elimina a necessidade de \textit{hardware} dedicado ou suporte técnico especializado para implementação.

\subsection{Desenvolver uma interface responsiva}

Uma das principais metas do DDR foi garantir que a interface de usuário fosse responsiva, intuitiva e amigável \cite{frontendperfomance}. Isso significa que a plataforma foi projetada para proporcionar uma experiência de uso fluida, tanto para operadores industriais quanto para engenheiros de processos, ao modelar fluxos e simulações de processos industriais \cite{reactjs}. A interface desenvolvida em \textit{React.js} permite a manipulação visual de processos, utilizando gráficos e fluxogramas que facilitam a compreensão dos dados e das variáveis envolvidas \cite{eloquentjavascript}. A criação de uma experiência de usuário simples e intuitiva foi fundamental para garantir que a plataforma fosse amplamente adotada, mesmo por usuários com pouca experiência prévia em sistemas de reconciliação de dados.

\subsection{Desenvolver a reconciliação de dados em bateladas}

O DDR foi desenvolvido com a capacidade de realizar reconciliação de dados em bateladas \cite{reconset}. Essa funcionalidade é essencial para ambientes industriais que operam em ciclos discretos, onde os dados podem ser coletados e reconciliados ao final de cada ciclo de produção \cite{danielhoduin}. A reconciliação em bateladas permite a correção e o ajuste dos dados medidos após o término de uma operação, garantindo que os resultados respeitem as leis de conservação de massa e energia \cite{balancecontrol}. Este tipo de reconciliação é particularmente útil para plantas industriais que operam com grandes volumes de dados acumulados durante um intervalo de tempo, permitindo uma análise detalhada e precisa dos dados ao final de cada batelada \cite{datarecshakar}.

\subsection{Desenvolver a reconciliação de dados em tempo real}

Além da reconciliação em bateladas, o DDR também foi projetado para realizar reconciliação de dados em tempo real \cite{datarecshakar}. Essa funcionalidade é especialmente relevante para plantas industriais que operam continuamente, exigindo correções imediatas nos dados coletados, permitindo uma resposta rápida às variações do processo \cite{danielhoduin}. A reconciliação em tempo real ajusta os dados à medida que são coletados, corrigindo erros de medição conforme ocorrem, aumentando assim a precisão dos dados usados para controle e monitoramento dos processos em tempo real \cite{reformulationdatarecon}. Essa capacidade é crucial para garantir a eficiência operacional e evitar decisões baseadas em dados imprecisos, aprimorando o controle dos processos industriais e otimizando o desempenho geral da planta \cite{computecontrol}.

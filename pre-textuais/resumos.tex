\begin{center}
    \LARGE PROPOSTA DE PROJETO DE PESQUISA 
    \normalsize 
\end{center}

\vspace{1cm}

\textbf{Titulo do Projeto:} "DDR - DASHBOARD OF DATA RECONCILIATION"

\vspace{1cm}

\textbf{Linha de Pesquisa:} Computalção

\vspace{1cm}

\textbf{Equipe Técnica:}

\vspace{0.5cm}

\textbf{Coordenador de Pesquisa:} João Gustavo Coelho Pena

\vspace{0.5cm}

\textbf{Orientadores:} 

- João Gustavo Coelho Pena

- Gustavo Post Sabin

\vspace{0.5cm}

\textbf{Orientandos (estudantes de graduaçõ):} 

- Nilton Aguiar dos Santos (RGA: 201811901022) - curso Eng. de Computação 


\mychapterast{Resumo}

Este Trabalho trata-se do Relatório Final do Projeto de Pesquisa intitulado "Projeto DDR – DASHBOARD OF DATA RECONCILIATION", nele é descrito os principais fundamentos, informações, metodologias e resultados alcançados.  

O Projeto DDR é um \textit{software online} voltado para reconciliação e qualidade de dados, utilizando técnicas de minimização de funções multivariáveis pelo método dos multiplicadores de Lagrange. A solução prioriza uma abordagem baseada na \textit{web}, oferecendo ao usuário a capacidade de realizar a análise e reconciliação de dados de forma remota e eficiente, com foco nos conceitos computacionais modernos, para proporcionar uma experiência facilitada. O \textit{software} aplica cálculos matemáticos e estatísticos ao longo de todo o processo, especificamente voltados para problemas de reconciliação de dados.

Por meio do DDR, é possível modelar todo um processo industrial, alimentá-lo com dados oriundos da planta em questão e, a partir disso, reconciliar e verificar a qualidade dos dados em tempo real. O \textit{software} se destaca como uma solução inovadora, pois não há concorrente direto que ofereça as mesmas funcionalidades em um ambiente mais acessível e ágil no estado de Mato Grosso. Ao longo deste trabalho, são detalhados o processo filosófico de desenvolvimento, os cálculos matemáticos, os conceitos estatísticos e computacionais, e a lógica do \textit{software} aplicada como solução. Exemplos do código funcional e considerações finais sobre o trabalho realizado são apresentados nos capítulos finais.

\vspace{1.5ex}

{\bf Palavras-chave}: CSS; \textit{Dashboard}; Desenvolvimento \textit{Web}; Desenvolvimento de \textit{Software}; HTML; Indústria 4.0; Javascript; Multiplicadores de Lagrange; Qualidade de Dados; Python; Reconciliação de Dados.


